\documentclass{rpt}

\begin{document}
\subsubsection*{Questions on \textit{mfs}}

\begin{enumerate}
\item Is \textit{mfs} a state-of-the-art method for logic synthesis with EXDCs?
Is there any other method to simplify with EXDCs?
\item Is \textit{mfs} only designed for FPGAs? Does it also work for ASICs?
\end{enumerate}

\subsubsection*{Questions on my research with \textit{mfs}}

I want to generate an approximate circuit by approximating its EXDCs.
I start from the \textit{mfs} method in the paper \textit{Scalable Don't-Care-Based Logic Optimization and Resynthesis}.
The {\bf only} change is the structure of miter in Figure~\ref{fig:ori} (the Figure 4.2.1 in the \textit{mfs} paper).

\begin{figure}[!htbp]
\centering
\includegraphics[scale = 0.2]{./sat.png}
\caption{Original miter for care set computation}\label{fig:ori}
\end{figure}

Instead of constructing such a miter in Figure~\ref{fig:ori} to represent the care set,
I simulate the circuit for $N$ rounds and generate a set of patterns for the window inputs $\mathbf{x}$.
Those patterns are treated as the {\bf approximate care set} and used to synthesize the approximate circuit.

For example, the window inputs are $A$, $B$ and $C$, and three rounds simulation patterns for $ABC$ are 010, 111, 101.
Then I use $O(x) = \overline AB \overline C + ABC + A \overline B C$ to represent the approximate care set.

\begin{enumerate}
\item Is it reasonable to change care set on window inputs?
\item I have an interesting observation on \textit{c880} with my idea:

In most cases, for some values of $N$ within a continuous interval,
the final approximate circuit does not change.
For example,
I increase $N$ from 192 to 256, the results are same.

But the circuit might be simplified dramatically due to the addition of one simulation frame.
For example, when $N$ change from 256 to 257 (the first 256 inputs patterns for simulation are same), the error rate of the final circuit changes greatly (from 0.2\% to 5\%).

Is the observation caused by the property of \textit{mfs}?
\end{enumerate}

\end{document}
